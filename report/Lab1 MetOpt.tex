\documentclass[a4paper, 14pt]{article}


\usepackage{cmap}
\usepackage[T2A]{fontenc}
\usepackage[utf8]{inputenc}
\usepackage[english,russian]{babel}
\usepackage{amsmath}
\usepackage{amsfonts}
\usepackage{amsmath,amsthm,amssymb}
\usepackage{color}
\usepackage{pgfplots}
\pgfplotsset{compat=1.9}



\begin{document}

	\renewcommand{\chaptername}{Лабораторная работа}
	\def\contentsname{Содержание}

	\begin{titlepage}
		\begin{center}
			\textsc{<<НАЦИАНАЛЬНЫЙ ИССЛЕДВАТЕЛЬСКИЙ УНИВЕРСИТЕТ ИТМО">\\[5mm]
			Факультет информационных технологий и программирования\\[2mm]
			Кафедра компьютерных технологий}

			\vfill

			\textbf{ОТЧЁТ ПО ЛАБОРАТОРНОЙ РАБОТЕ №1\\[3mm]
			Методы одномерной оптимизации\\[6mm]
			Вариант 4
			\\[20mm]
			}
		\end{center}

		\hfill
		\begin{minipage}{.5\textwidth}
			Выполнили студенты:\\[2mm]
			Ефимов Сергей Алексеевич\\
			группа: М3237\\[2mm]
			Соколов Александр Андреевич\\
			группа: М3234\\[5mm]

			Проверил:\\[2mm]
			доцент кафедры суетологии\\
			Пупкина Залупкина
		\end{minipage}%
		\vfill
		\begin{center}
			г. Санкт-Петербург
		\end{center}
	\end{titlepage}


	\section*{Постановка задачи}
	Задача лаборатрной работы  -- научиться реализовывать алгоритмы одномерной минимизации функции каждым из следуших способов:
	\begin{enumerate}
		\item Метод дихотомии
		\item Метод золотого сечения
		\item Метод Фиббоначи
		\item Метод парабол
		\item Комбинированный метод Брента
	\end{enumerate}
	Также необходимо решить задачу аналитически и привести отчет по результатам сравнения проведенных вычислений
	\section*{Ход работы}
	Исследуемая функция \[f(x) = x -   \ln(x)\] Интервал исследования [0.5, 4]
	\subsection*{1. Аналитическое решение}
	Для наглядности приведем график функции на заданном интервале:

	\begin{tikzpicture}
		\begin{axis}[
		title = x - ln(x),
		xlabel = {$x$},
		ylabel = {$y$},
		minor tick num = 2,
		xmin = 0.5,
		xmax = 4
		]
		\addplot[red] {x - ln(x)};
		\end{axis}
	\end{tikzpicture}

	Чтобы найти $/min$ функции необходимо найти ее критические точки(точки, в которых производная функции равна нуля $f'(x) = 0$):
	\[f'(x) = (x - \ln(x))' = x' - \ln(x)' = 0\]
	\[1 - \frac{1}{x} = 0 \Rightarrow x = 1\]
	Итак, мы нашли критическую точку, теперь осталось сравнить три значения функции -- в критической точке и в точках, являющиеся концами исследуемного отрезка. Пусть $x$ критическая точка, $a, b$ -начало и конец отрезка соответственно:
	\[f(a) = 0.5 - \ln(0.5) \approx 0.5 + 0.6931  \approx 1.1931\]
	\[f(b) = 4 - \ln(4) \approx 4 - 1.3863 \approx 2.6137\]
	\[f(x) = 1 - \ln(1) = 1\]
	Из приведенных выше вычислений можно утверждать, что $\min$ функции $f(x) = x - \ln(x)$ равен 1, при $x =1$, (точка $M(1,1)$)
	\subsection*{Метод дихотомии}
	Давайте рассмотрим на примере нашей функции метод дихотомии, который основывается на методе приближений, сокращая интревал поиска таким образом. что сохраняется инвариант - с каждой интерацией отрезок сокращается, минимум находится в пределах отрезка.

	Написав код к решению задачи данным способом(который можно увидеть ниже), мы имеем возможность привести таблицу выполнения данного алгоритма:\\\\
	\begin{tabular}{|c|c|c|}
		\hline
		Текущий интервал & Соотношение отрезка с предыдущей итерацией & текущий минимум \\
		\hline
		[0.5, 4] & - & - \\
		\hline
		[0.5, 2.25] & 1.999 & 1.375 \\
		\hline
		[0.5 1.375] & 1.999 & 0.9375 \\
		\hline
		[0.973, 1.375] & 1.999 & 1.156 \\
		\hline
		[0.973, 1.156] & 1.999 & 1.0468 \\
		\hline
		[0.973, 1.0468] & 1.999 & 0.992 \\
		\hline
		[0.992, 1.0468] & 1.999 & 1.019 \\
		\hline
		[0.992, 1.019] & 1.999 & 1.005 \\
		\hline
		[0.992, 1.005] & 1.999 & 0.999 \\
		\hline
		[0.999, 1.005] & 1.999 & 1.002 \\
		\hline
		[0.999, 1.002] & 1.999 & 1.001 \\
		\hline
		[0.999, 1.001] & 1.999 & 1.000 \\
		\hline
	\end{tabular} \\

	Исходя из данных таблицы можно утверждать, что каждую итерацию алгоритм сужает область поиска $\approx$ в 2 раза, что говорит о линейной сходимости данного алгоритма. Из этих соображений вытекает формула
	\[|x_n - x_{min}| \leqslant \frac{1}{2^n} \cdot (b - a)\]

	Чтобы оценить метод дихотомии давайте построим график зависимости кол-ва вычислений функции от логарифма точности:


	\begin{tikzpicture}
		\begin{axis}[
		title = График зависимости кол-ва вычислений функции от логарифма точности,
		xlabel = {$\ln(\epsilon)$},
		ylabel = {Кол-во вычислений}
		]
		\addplot coordinates {
			( -6.907, 11 )
			( -9.21, 15 )
			( -11.513, 18 )
			( -13.816, 21 )
			( -16.118, 25 )
			( -18.42, 28 )};
		\end{axis}
	\end{tikzpicture}

	Исходя из графика можно говорить о том, что кол-во вычислений данного метода линейно зависит от $\ln(\epsilon)$ причем чем больше $\ln(\epsilon)$, тем меньше кол-во вычислений, иными словами алгоритм делает больше итераций, если задаваемая точность выше, что логично


	Теперь рассмотрим алгоритм на примере многомодальной функции. Возьмем функцию:
	\begin{equation*}
		f(x) =
		\begin{cases}
			x < 1, f(x) = |x|\\
			x \eqslantgtr 1, f(x) = |x - 4| - 2
		\end{cases}
	\end{equation*}

	Ее график:
	\\

	\begin{tikzpicture}
		\begin{axis}[
			title = График приведенной нами многомодальной функции:,
			xlabel = x,
			ylabel = y
		]
			\addplot coordinates {
				( -3, 3 )
				( -2, 2 )
				( -1, 1 )
				( 0, 0 )
				( 1, 1 )
				( 2, 0 )
				( 3, -1 )
				( 4, -2 )
				( 5, -1 )
				( 6, 0 )
				( 7, 1 )};
		\end{axis}
	\end{tikzpicture}


	Алгоритм выдает $x = 0$, как минимум всей функции, но как видно по графику, приведенному выше, минимум функции на отрезке $[-3, 7]$ достигается в точке $x = 4$, откуда следует вывод о том, что данный метод НЕ работает на многомодальных функциях. Однако стоит заметить, что данный алгоритм всегда находит минимум(локальный или глобальный)
\end{document}
