\documentclass[a4paper, 42pt]{article}


\usepackage{cmap}
\usepackage[T2A]{fontenc}
\usepackage[utf8]{inputenc}
\usepackage[english,russian]{babel}
\usepackage{amsmath}
\usepackage{amsfonts}
\usepackage{amsmath,amsthm,amssymb}
\usepackage{color}
\usepackage{pgfplots}
\pgfplotsset{compat=1.9}



\begin{document}

	\renewcommand{\chaptername}{Лабораторная работа}
	\def\contentsname{Содержание}

	\begin{titlepage}
		\begin{center}
			\textsc{<<НАЦИАНАЛЬНЫЙ ИССЛЕДВАТЕЛЬСКИЙ УНИВЕРСИТЕТ ИТМО">\\[5mm]
				Факультет информационных технологий и программирования\\[2mm]
				Кафедра компьютерных технологий}

			\vfill

			\textbf{ОТЧЁТ ПО ЛАБОРАТОРНОЙ РАБОТЕ №1\\[3mm]
				Методы одномерной оптимизации\\[6mm]
				Вариант 4
				\\[20mm]
			}
		\end{center}

		\hfill
		\begin{minipage}{.5\textwidth}
			Выполнили студенты:\\[2mm]
			Ефимов Сергей Алексеевич\\
			группа: М3237\\[2mm]
			Соколов Александр Андреевич\\
			группа: М3234\\[5mm]

			Проверил:\\[2mm]
			доцент кафедры суетологии\\
			Пупкина Залупкина
		\end{minipage}%
		\vfill
		\begin{center}
			г. Санкт-Петербург
		\end{center}
	\end{titlepage}


	\section*{Постановка задачи}
	Задача лаборатрной работы  -- научиться реализовывать алгоритмы одномерной минимизации функции каждым из следуших способов:
	\begin{enumerate}
		\item Метод дихотомии
		\item Метод золотого сечения
		\item Метод Фиббоначи
		\item Метод парабол
		\item Комбинированный метод Брента
	\end{enumerate}
	Также необходимо решить задачу аналитически и привести отчет по результатам сравнения проведенных вычислений
	\section*{Ход работы}
	Исследуемая функция \[f(x) = x -   \ln(x)\] Интервал исследования [0.5, 4]
	\subsection*{1. Аналитическое решение}
	Для наглядности приведем график функции на заданном интервале:

	\begin{tikzpicture}
		\begin{axis}[
			title = x - ln(x),
			xlabel = {$x$},
			ylabel = {$y$},
			minor tick num = 2,
			xmin = 0.5,
			xmax = 4
		]
			\addplot[red] {x - ln(x)};
		\end{axis}
	\end{tikzpicture}

	Чтобы найти $/min$ функции необходимо найти ее критические точки(точки, в которых производная функции равна нуля $f'(x) = 0$):
	\[f'(x) = (x - \ln(x))' = x' - \ln(x)' = 0\]
	\[1 - \frac{1}{x} = 0 \Rightarrow x = 1\]
	Итак, мы нашли критическую точку, теперь осталось сравнить три значения функции -- в критической точке и в точках, являющиеся концами исследуемного отрезка. Пусть $x$ критическая точка, $a, b$ -начало и конец отрезка соответственно:
	\[f(a) = 0.5 - \ln(0.5) \approx 0.5 + 0.6931  \approx 1.1931\]
	\[f(b) = 4 - \ln(4) \approx 4 - 1.3863 \approx 2.6137\]
	\[f(x) = 1 - \ln(1) = 1\]
	Из приведенных выше вычислений можно утверждать, что $\min$ функции $f(x) = x - \ln(x)$ равен 1, при $x =1$, (точка $M(1,1)$)


\end{document}
